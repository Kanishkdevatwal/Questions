\let\negmedspace\undefined
\let\negthickspace\undefined
\documentclass[journal,12pt,onecolumn]{IEEEtran}
\usepackage{cite}
\usepackage{amsmath,amssymb,amsfonts,amsthm}
\usepackage{algorithmic}
\usepackage{graphicx}
\usepackage{textcomp}
\usepackage{xcolor}
\usepackage{txfonts}
\usepackage{listings}
\usepackage{enumitem}
\usepackage{mathtools}
\usepackage{gensymb}
\usepackage{comment}
\usepackage[breaklinks=true]{hyperref}
\usepackage{tkz-euclide} 
\usepackage{listings}
\usepackage{gvv}                                        
\def\inputGnumericTable{}                                 
\usepackage[latin1]{inputenc}                                
\usepackage{color}                                            
\usepackage{array}                                            
\usepackage{longtable}                                       
\usepackage{calc}                                             
\usepackage{multirow}                                         
\usepackage{hhline}                                           
\usepackage{ifthen}                                           
\usepackage{lscape}
\usepackage{textcomp}

\newtheorem{theorem}{Theorem}[section]
\newtheorem{problem}{Problem}
\newtheorem{proposition}{Proposition}[section]
\newtheorem{lemma}{Lemma}[section]
\newtheorem{corollary}[theorem]{Corollary}
\newtheorem{example}{Example}[section]
\newtheorem{definition}[problem]{Definition}
\newcommand{\BEQA}{\begin{eqnarray}}
\newcommand{\EEQA}{\end{eqnarray}}
\newcommand{\define}{\stackrel{\triangle}{=}}
\theoremstyle{remark}
\newtheorem{rem}{Remark}
\begin{document}

\bibliographystyle{IEEEtran}
\vspace{3cm}

\title{ec23...}
\author{EE23BTECH11029 - Kanishk}
\maketitle

\bigskip

\renewcommand{\thefigure}{\theenumi}
\renewcommand{\thetable}{\theenumi}
\textbf{Question}:\\ 
In the table shown below, match the signal type with its spectral characteristics.\hfill{GATE 2023 EC}

\vspace{2mm}

\begin{table}[ht]
    \centering
    \def\arraystretch{2}
    \begin{tabular}{|c|c|}
\hline
Signal Type & Spectral Characterstics \\
\hline
$\brak{i}$  Continuous, aperiodic & \brak{a} Continuous, aperiodic\\
\hline
$\brak{ii}$ Continuous, periodic & \brak{b} Continuous, periodic \\
\hline
$\brak{iii}$ Discrete, aperiodic & \brak{c} Discrete, aperiodic\\
\hline
$\brak{iii}$ Discrete, periodic & \brak{c} Discrete, periodic\\
\hline
\end{tabular}
    \caption{ }
    \label{29.2023}
\end{table}

\begin{enumerate}
\item \brak{\romannumeral 1} \textrightarrow \brak{a}  ,   \brak{\romannumeral 2} \textrightarrow \brak{b}   ,   \brak{\romannumeral 3} \textrightarrow \brak{c}   ,   \brak{\romannumeral 4} \textrightarrow \brak{d}

\item \brak{\romannumeral 1} \textrightarrow \brak{a}   ,  \brak{\romannumeral 2} \textrightarrow \brak{c}   ,   \brak{\romannumeral 3} \textrightarrow \brak{b}   ,   \brak{\romannumeral 4} \textrightarrow \brak{d}

\item  \brak{\romannumeral 1} \textrightarrow \brak{d}   ,   \brak{\romannumeral 2} \textrightarrow \brak{b}   ,   \brak{\romannumeral 3} \textrightarrow \brak{c}   ,   \brak{\romannumeral 4} \textrightarrow \brak{a}

\item \brak{\romannumeral 1} \textrightarrow \brak{a}   ,  \brak{\romannumeral 2} \textrightarrow \brak{c}    ,   \brak{\romannumeral 3} \textrightarrow \brak{d}   ,  \brak{\romannumeral 4} \textrightarrow \brak{b}

\end{enumerate}

\textbf{Solution}:\\


\begin{table}[ht]
    \centering
    \def\arraystretch{2}
    \begin{tabular}{|c|c|}
\hline
Parameter & Description \\
\hline
$x\brak{t}$  & Continuous Time  Signal\\
\hline
$x\brak{f}$ & Fourier Transform of a Signal \\
\hline
$x\sbrak{n}$ & Discrete Time Signal\\
\hline
$X\sbrak{k}$ &  Tthe amplitude and phase of the $k^{th}$frequency component of the input signal $x\sbrak{n}$\\
\hline
\end{tabular}
    \caption{Input Parameters}
    \label{1}
\end{table}

\newpage
\textbf{1. Continuous, aperiodic signal}\\

\begin{align}
X\brak{f}&=\int_{-\infty}^{\infty} x\brak{t}.e^{-j2\pi ft}\,dt
\end{align}

Let's consider the limit as $f$ approaches a certain frequency $f_0$:

\begin{align}
lim_{\epsilon \rightarrow 0}X\brak{f_0+\epsilon}&= lim_{\epsilon \rightarrow 0}\int_{-\infty}^{\infty} x\brak{t}.e^{-j2\pi ft}\,dt
\end{align}

By continuity of $x\brak{t}$ we can interchange the limit and the integral:
\begin{align}
&=\int_{-\infty}^{\infty} x\brak{t}.lim_{\epsilon\rightarrow0}e^{-j2\pi\brak{f_0+\epsilon}t}\,dt\\
&=\int_{-\infty}^{\infty} x\brak{t}.e^{-j2\pi f_0 t}\,dt\\
&=X\brak{f_0}\\
 lim_{f_0-\epsilon}&=X\brak{f_0}
\end{align}

Therefore, X\brak{f} is continuous for all frequencies $f$.

Let's assume $X\brak{f}$ is periodic with period $T$ 

\begin{align}
X\brak{f+T}&=X\brak{f}
\end{align}
Now applying inverse Fourier transform

\begin{align}
x\brak{t}&=\int_{-\infty}^{\infty} X\brak{f}.e^{j2\pi ft}\,df\\
&=\int_{-\infty}^{\infty} X\brak{f+T}.e^{j2\pi \brak{f+T}t}\,df\\
&=\int_{-\infty}^{\infty} X\brak{f}.e^{j2\pi \brak{f+T}t}\,df\\
&=e^{j2\pi Tt}\int_{-\infty}^{\infty}.X\brak{f}.e^{2\pi ft}\, df
\end{align}

$e^{j2\pi Tt}$ is a periodic function of $t$, which contradicts that $x\brak{t}$ is aperiodic,

Therefore, $X\brak{f}$ cannot be periodic, hence it must be aperiodic.\\
\vspace{2mm}


For Example: Exponential Decay\\

From \tabref{1}
\begin{align}
x\brak{t}&=e^{-2 t}.u\brak{t}\\
X\brak{f}&= \int_{-\infty}^{\infty} e^{-2ft}u\brak{t}.e^{-j2\pi ft}\,dt\\
&= \int_{0}^{\infty} e^{-\brak{2+j2\pi f}t}\,dt\\
&= \frac{1}{2\brak{1 + j\pi f}}
\end{align}

$\implies X\brak{f} $ is continuous and aperiodic \\
 

\textbf{2. Continuous, periodic signal}\\

\begin{align}
x\brak{t}&=x\brak{t+kT}\\
x\brak{t}& \leftrightarrow X\brak{f}\\
X\brak{f}&=\int_{-\infty}^{\infty} x\brak{t}.e^{-j2\pi ft}\,dt\\
X\brak{f}&=\int_{-\infty}^{\infty} x\brak{t+kT}.e^{-j2\pi ft}\,dt\\
\text{Let } \tau &= t+kT\\
X\brak{f}&=\int_{-\infty}^{\infty} x\brak{\tau}.e^{-j2\pi f\tau}.e^{j2\pi fkT}\,d\tau
\end{align}

$e^{j2\pi fkT} $ is a Periodic function in frequency domain with period $\frac{1}{T}$ Hz \\
\begin{align}
X\brak{f}&=e^{j2\pi fkT}\int_{-\infty}^{\infty} x\brak{\tau}.e^{-j2\pi f\tau}\,d\tau\\
\implies &X\brak{f} \text{ is aperiodic}
\end{align}


$X\brak{f}$ will be  a sum of scaled copies of the Fourier transform of $x\brak{t}$, each copy shifted by multiples of $\frac{1}{T}$Hz This results in a discrete frequency domain representation.\\
\vspace{2mm}

For example: Sine wave function\\

From \tabref{1}
\begin{align}
x\brak{t}&=sin\brak{2\pi ft}\\
sin\brak{f_0t}&= \frac{1}{2j}\sbrak{e^{j2\pi f_0 t} -  e^{-j2\pi f_0 t} }\\
X\brak{f} &=  \frac{1}{2j}  \sbrak{2\pi \delta\brak{2\pi f- 2\pi f_0} - 2\pi \delta\brak{2\pi f+2\pi f_0}  }\\
&= j\pi \sbrak{\delta\brak{2\pi f + 2\pi f_0} -   \delta\brak{2\pi f - 2\pi f_0}}
\end{align}

$\implies X\brak{f}$ is Discrete and Aperiodic Signal\\


\textbf{3. Discrete, aperiodic signal}\\

\begin{align}
X\brak{e^{j2\pi f}}&= \sum_{n=-\infty}^{\infty} x\sbrak{n}.e^{-j2\pi fn}
\end{align}


Let's consider the Discrete time Fourier Transform (DFT) of $x\sbrak{n}$, denoted by $ X\sbrak{k}$ which is defined as:

\begin{align}
X\sbrak{k}&=\sum_{n=0}^{N-1}x\sbrak{n}e^{-j\frac{2\pi}{N}kn}
\end{align}
The DFT is inherently periodic in frequency domain with period $\frac{1}{N}$.\\


As  $N$ approaches infinity, the spacing between frequency samples in the frequency domain  $\frac{1}{N}$  approaches zero. Therefore, the DFT $X\sbrak{k}$ approaches the continuous Fourier Transform $X\brak{f}$. \\

Since the DFT $X\brak{f}$  is periodic in the frequency domain, its limiting form $X\brak{f}$ is also periodic, continuous in frequency domain.\\

For example: Exponential Decay\\

From \tabref{1}

\begin{align}
x\sbrak{n}&= e^{n}.u\sbrak{n}\\
X\brak{e^{j2\pi f}}&= \sum_{n=-\infty}^{\infty} e^{n}.u\sbrak{n}.e^{-j2\pi fn}\\
&=  \sum_{n=0}^{\infty} e^{\brak{1-j2\pi f}^{n}}\\
&=\frac{1}{1-e^{\brak{1-j2\pi f}}}
\end{align}
$\implies $ Continuous and Periodic signal\\

\vspace{2mm}
\textbf{4. Discrete,periodic signal}

\begin{align}
X\brak{f}&= \sum_{n=-\infty}^{\infty} x\sbrak{n}.e^{-j2\pi fn}
\end{align}

The Discrete Fourier Transform(DFT) is periodic in frequency domain with period $\frac{1}{N}$\\

As $N$ is finite , Therefore DFT has finite period $\frac{1}{N}$

\begin{align}
X\sbrak{k}&=\sum_{n=0}^{N-1}x\sbrak{n}e^{-j\frac{2\pi}{N}kn}
\end{align}
The DFT computation involves summing over a finite number of samples . Therefore, the resulting spectrum  $X\sbrak{k}$ is inherently discrete.\\

For example:  Sinusoidal Signal\\

From \tabref{1}

\begin{align}
x\sbrak{n}&=sin\brak{2\pi f_0n}.u\sbrak{n}\\
X\brak{e^{j2\pi f}}&= \sum_{n=-\infty}^{\infty} sin\brak{2\pi f_0n}.u\sbrak{n}.e^{-j2\pi fn}\\
&= \sum_{n=0}^{\infty} \frac{e^{j2\pi f_0n}- e^{-j2\pi f_0n}}{2j}. e^{-j2\pi fn}\\
&=\frac{1}{2j}\sum_{n=0}^{\infty} \brak{e^{j\brak{2\pi f_0n-2\pi fn}} -  e^{-j\brak{2\pi f_0n+2\pi fn}} }\\
&= \frac{1}{2j} \brak{\frac{1}{1-e^{j\brak{2\pi f_0-2\pi f}}}  -  \frac{1}{1-e^{-j\brak{2\pi f_0+ 2\pi f}}}}
\end{align}
$\implies$Discrete and Periodic \\

\brak{\romannumeral 1} \textrightarrow \brak{a}   ,  \brak{\romannumeral 2} \textrightarrow \brak{c}   ,   \brak{\romannumeral 3} \textrightarrow \brak{b}   ,   \brak{\romannumeral 4} \textrightarrow \brak{d}

\end{document}
